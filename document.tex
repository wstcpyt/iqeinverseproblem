\documentclass[12pt]{article}
\usepackage{amssymb,amsmath,latexsym}

% Page length commands go here in the preamble
\setlength{\oddsidemargin}{-0.25in} % Left margin of 1 in + 0 in = 1 in
\setlength{\textwidth}{7in}   % Right margin of 8.5 in - 1 in - 6.5 in = 1 in
\setlength{\topmargin}{-.75in}  % Top margin of 2 in -0.75 in = 1 in
\setlength{\textheight}{9.2in}  % Lower margin of 11 in - 9 in - 1 in = 1 in

\newtheorem{theorem}{Theorem}
\newtheorem{definition}{Definition}

\renewcommand{\baselinestretch}{1.5} % 1.5 denotes double spacing. Changing it will change the spacing

\setlength{\parindent}{0in} 
\begin{document}
\title{Reconstruction of the charge collection probablity in a CIGS solar cell
from Internal Quantum Efficiency measurements by the regularization method}
\date{\today}
\maketitle
\abstract{A new method is introduced to analyze the Internal Quantum
Efficiency(IQE) data that are obtained on a $Cu(In,Ga)S{{e}_{2}}$(CIGS) solar
cells by irradiation
with variable wavelength. This method aims at directly reconstruction the
distribution $\phi(z)$ of the charge collection probability in the device, by
inverting the integral transform that connects $\phi(z)$ to the neasured IQE.
This inversion is generally an ill-posed problem, which, however can be solved
by the regularization method. This procedure is used to
analyze the simulated and experiment measurements IQE data on CIGS solar cells.}
\section{Introduction}
\section{Theory}
\subsection{Numerical results and discussion}

\end{document}